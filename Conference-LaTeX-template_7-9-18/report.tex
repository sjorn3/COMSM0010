\documentclass[conference]{IEEEtran}
\IEEEoverridecommandlockouts
% The preceding line is only needed to identify funding in the first footnote. If that is unneeded, please comment it out.
\usepackage{cite}
\usepackage{amsmath,amssymb,amsfonts}
\usepackage{algorithmic}
\usepackage{graphicx}
\usepackage{textcomp}
\usepackage{xcolor}
\def\BibTeX{{\rm B\kern-.05em{\sc i\kern-.025em b}\kern-.08em
  T\kern-.1667em\lower.7ex\hbox{E}\kern-.125emX}}
\begin{document}

\newcommand{\name}{Paper}

\title{\name{}: A Realtime Shared Whiteboard}
% {\footnotesize A realtime shared whiteboard.}

\author{\IEEEauthorblockN{Sean Innes}
\IEEEauthorblockA{\textit{University of Bristol} \\
sean.innes@bristol.ac.uk}
}

\maketitle

\begin{abstract}
In a personal setting, the ability to jot down ideas and share thoughts is
invaluable to both the creative and educative processes. On a global scale, the
ability to engage in these activities is dramatically reduced by the physical
distance between you. In a packed lecture hall setting, a lecturer can't do much
more than talk at their students and hope for the best. \name{} aims to provide
the shared scratch surface to explore and develop ideas in real time, and allow
engagement from all parties.
\end{abstract}

\section{Introduction}
// Ramble to express ideas, add proper structure in the final version.

Our idea is to create a real time shared whiteboard. The idea is simple enough,
but quite wide in scope, so our exact remit is restricted to a website that
gives a link to a whiteboard, and anyone who accesses this link is able to edit
content. There are countless ways this could be extended, some of which are
outlined below. The usage of such a page also potentially goes well beyond that
of a standard whiteboard. For example playing noughts and crosses with a friend
in the US, which is why we've gone for the more general name \name{}.

Since realtimeness and scalability are crucial to demonstrate this products
viability, the initial aim would be to get a working demonstration of this.
Showing that latency is respectable both when many users access the same
whiteboard and when many whiteboards exist.

\subsection{Extensions}

There are many ways this could be extended, some of our favourite ideas are
outlined here:

\begin{itemize}
  \item Allow submissions to be vetted by the creator of the whiteboard. Perhaps
  by their own decision or give the option for the group to vote. If a lecturer
  asks a question, some students could submit answers (as drawings on the
  whiteboard) and the rest could vote on the ones they think are correct, for
  example, and the best one can be added to the board (as decided by the owner).
  \item Since this is simple vector graphics, you can record their application
  to the board and allow later playback. Perhaps alongside a recorded video
  call or lecture.
  \item To monetise the project, a whiteboard could be restricted to ten users,
  and an appropriate license lifts this restriction. For example an
  organization or university license.
  \item Accessing the link after the presentation/discussion is over we can
  provide a pdf or similar of the final state of the board. Additionally an
  animation if that feature is implemented.
  \item Generally improved tooling. Multiple colours, adding images, shapes,
  perhaps a background pdf so you could draw on a presentation.
\end{itemize}

\subsection{Implementation}
Likely a Kubernetes cluster that stores the current state of the boards and can
report updates to this state to clients which subscribe to them. Preferably over
sockets due to the latency benefit. The frontend will be written in elm, a typed
functional language that compiles to javascript, since it is very efficient at
manipulating DOM and Canvas elements. 

In general, design decisions should have the central goal of minimizing the
latency between writing on the board and everyone with access to it seeing this
change. Ideally this latency would be comfortably below whatever the latency of
the video feed or other connection has.

\end{document}
